\documentclass[aspectratio=169,t]{beamer}
\usepackage[utf8]{inputenc}
\usepackage[T1]{fontenc}
\usepackage{comment}
\usepackage{listings}
% \lstset{basicstyle=\scriptsize}

\title[2022-05-27---CSIS Seminar---David Eyers]{More cloudy student teaching---now with IDEs}
\date{CSIS Seminar, 2022-05-27}
\author{\textbf{David Eyers}}

\usetheme{oucs1}

\begin{document}

\begin{frame}
  \titlepage
\end{frame}

\begin{dframe}{Motivation and plan}
  \1 Again talking about \textbf{cloud potential for teaching}\\
     \dots but \textbf{lots has changed} in the last two years, e.g.,

  \2 I now edit \LaTeX{} using Visual Studio Code instead of Emacs
  \2 I finally converted Lech's lecture slide template to Beamer
  
  \bigskip

  \1 My plan is to cover two topics:

  \2 Using \textbf{continuous integration} for teaching material
  \2 Potentially adopting \textbf{cloud-based IDEs} for teaching

  \bigskip

  \1 Also aim to explain relevance of two changes above!

\end{dframe}

\begin{dframe}{Outgoing CS webserver setup}
  \1 Previously: all CS course material at \url{www.cs.otago.ac.nz}

  \2 Content normally reaches the web care of \url{prism.otago.ac.nz}
  \2 Content usually publicly available---good!
  \2 Supports dynamic web page content, e.g., using PHP
  
  \3 Security isolation through server design

  \medskip

  \1 Pandemic led to CS shift to Blackboard

  \2 Aim to simplify student experience---important
  \2 Blackboard handles some student/teaching needs OK (?)
  \2 Soon: Blackboard will move to cloud; also LMS review
\end{dframe}

\begin{dframe}{Web technology evolution}
  \1 Typically don't need PHP for simple content hosting

  \2 Dynamic web languages need server execution
  
  \3 Security and maintenance hassle...

  \1 Now common to seek static content

  \2 Works well with CDNs
  \2 JavaScript is typically assumed for client-side Automation
  \2 Can include dynamic content using \texttt{iframe}s

  \1 Need something powerful? Grab yourself a virtual machine, or a container
\end{dframe}

\begin{dframe}{SSG---Static Site Generaors}
  \1 Move template generation back to compile time

  \2 Run SSG over source code, HTML files produced
  \2 Often Markdown used for source code
  \2 Causes redundant HTML, but not a resource challenge

  \1 Popularised through GitHub pages

  \2 Web serving becomes simple and easy

  % \3 ... which is why GitHub can do it for free

  \1 Many SSG tools exist, for example:

  \2 Jekyll---written in Ruby; used by GitHub Pages

  \2 Hugo---written in Go
\end{dframe}

\begin{dframe}{CI---Continuous Integration}
  \1 SSG tools work well with continuous integration

  \2 You update your source content, commit, push
  \2 CI runs SSG to build HTML
  \2 Install HTML on web server

  \bigskip
  
  \1 This is covered in COSC212 and COSC202
  
  \bigskip

  \1 Demo-wise, let's create the website for a new paper
  
  \2 We'll use GitLab and the Hugo SSG
  % TODO template creation of COSC312
\end{dframe}

\begin{dframe}{Jekyll builds Carpentries teaching material}
  \1 The Carpentries promotes digital research skills

  \2 Releases materials as open source

  \bigskip
  
  \1 Runs an incubator for new lesson material
  
  \2 I contributed the Docker lesson
  \2 Have also used in COSC349 and COSC202
  
  \bigskip

  \1 Let's build that lesson using Docker
%  \1 But now can move this entirely into cloud
\end{dframe}

\begin{dframe}{Visual Studio Code (VSCode)}
  \1 Wanted to move COSC202 to IDE usable broadly
  \1 VSCode development speed has been insanely fast:
  
  \2 compositionality of extensions is extremely impressive
  \2 e.g., I use a code spellcheck extension to spellcheck my \LaTeX{}

  \1 Integrates version control straightforwardly
  
  \2 thus supports Git-driven continuous integration
  
  \1 Many more tricks up its sleeve, since it's an electron application...
\end{dframe}

\begin{dframe}{VSCode and Electron applications}
  \1 Electron applications use a browser to do rendering
  
  \2 Also means they're close to hosting on web

  \bigskip

  \1 Refactoring of VSCode now provides VSCode Server
  
  \2 i.e., you can access VSCode remotely via a web browser
  
  \bigskip

  \1 VSCode already integrated the idea of remote development environments

  \2 e.g., development within software containers, over SSH, etc.
\end{dframe}

\begin{dframe}{Codespaces}
  \1 GitHub Codespaces launched in beta in 2020 % TODO

  \2 Provides a means to manage development environments around repositories
  \2 Active rather than the scheduled feel of continuous integration
  \2 but can commit back to repositories to trigger CI

  \bigskip

  \1 Let's use Codespaces to edit the Docker lab
\end{dframe}

\begin{dframe}{BYOD and remote teaching}
 % \1 So we get to the teaching relevance
  \1 Advantage of cloud-based IDEs include:
  
  \2 Can be hosted in the cloud, or onsite, or both
  \2 Students can access the environment on any device
  \2 CS Lab environment or remote: consistent access to software

  \1 Key challenge: GUIs
  
  \2 Not impossible to handle, but typically web is targetted
%  \2 and that doesn't suit all cases

  \1 Very keen to test the practicality of this
  
  \2 May couple well with the CS lab shift to Windows for 2023
\end{dframe}

\begin{dframe}{Containers that build \LaTeX{} into PDF}
  % \1 Static websites for openly sharing open source lecture material is a bit niche

  \1 Using \LaTeX{} to make your lecture notes?
  
  \2 Can use CI to share with students the PDFs produced
  \2 Can create Docker container that builds \LaTeX{}
  
  \1 Using VSCode to edit your \LaTeX{}

  \2 Can use online VSCode to update your notes from anywhere

  % \2 Automation efficiency: want compilation of notes to release them on the web too

  \1 Support Blackboard by including PDFs by URL
 
  \2 URLs from Blackboard link to CI-generated PDFs.

  \2 (or just link out to your own static website...)

  % \1 Discuss creating a container that builds content using \LaTeX{}
\end{dframe}

% include SOAP?

\begin{dframe}{Gitpod}
  \1 Codespaces is great, but proprietary
  \1 Gitpod is to Codespaces what GitLab is to GitHub
  
  \2 open source implementation of cloud IDE
  \2 also a hosted service in the cloud

  \1 Originally Gitpod used a VSCode-like editor

  \2 however now `everyone' is using VSCode
  \2 So Gitpod does too
  \2 Gitpod was key in development of VSCode Server
\end{dframe}

\begin{dframe}{Git-based web can share large files (LFS)}
  \1 Many ways to share files publicly of different sizes

  \2 e.g., CloudStor, OneDrive(ish), FileDrop, Globus, etc.
  \2 (ITS can advise on the different options)
%  \2 sharing seems to only be for limited times only

  \1 CS website allowed persistent public sharing of files

  \2 but adding transient files to Git not ideal
  \2 `deleted' files will still be in Git history

  \1 GitLab supports Git LFS---i.e., large file support

  \2 Files in Git are instead more like symlinks
  \2 Git LFS organises for large files to be downloaded

  % example of using LFS to share files

\end{dframe}

\begin{comment}
\begin{dframe}{FaaS, serverless and responsiveness}
  \1 Edge appearance
\end{dframe}
\end{comment}

\end{document}

Experimentation

https://gitpod.io/#https://github.com/dme26/2022-CSIS-Seminar

https://www.gitpod.io/docs/config-gitpod-file
gp init # in Gitpod shell

add into .gitpod.yml
image: dme26/latex-builder:bullseye-slim

12:01:30 started building image

Better to use the Gitpod project idea, and install the packages directly

Setup on Gitpod
Building TeXLive

Started 12:49:30
Finished 12:55:30

Install vscode-pdf

latexmk -norc -r latexmk-ci-config -pvc -view=none

\documentclass[aspectratio=169,t]{beamer}
\usepackage[utf8]{inputenc}
\usepackage[T1]{fontenc}
\usepackage{comment}
\usepackage{listings}
% \lstset{basicstyle=\scriptsize}

\title[2022-05-27---CSIS Seminar---David Eyers]{More cloudy student teaching---now with IDEs}
\date{CSIS Seminar, 2022-05-27}
\author{\textbf{David Eyers}}

\usetheme{oucs1}

\begin{document}

\begin{frame}
  \titlepage
\end{frame}

\begin{dframe}{Motivation and plan}
  \1 Again talking about \textbf{cloud potential for teaching}\\
     \dots but \textbf{lots has changed} in the last two years, \emph{e.g.},

  \2 I now edit \LaTeX{} using Visual Studio Code instead of Emacs
  \2 I finally converted Lech's lecture slide template to Beamer
  
  \medskip

  \1 My plan is to cover two topics:

  \2 Using \textbf{continuous integration} for teaching material
  \2 Potentially adopting \textbf{cloud-based IDEs} for teaching

  \medskip

  \1 Also aim to explain relevance of two changes above!

\end{dframe}

\begin{dframe}{Outgoing CS webserver setup}
  \1 Previously: all \textbf{CS course material} at \url{www.cs.otago.ac.nz}

  \2 Content normally reaches the web care of \url{prism.otago.ac.nz}
  \2 Content usually \textbf{was publicly available}---open is good!
  \2 Supports dynamic web page content, \emph{e.g.}, using PHP
  
  \3 Security isolation through server design

  \medskip

  \1 COVID-19 pandemic led to CS \textbf{shift to Blackboard}

  \2 Aim to \textbf{simplify student experience}---important
  \2 Blackboard handles some student/teaching needs OK (?)
  \2 Soon: Blackboard will \textbf{move to cloud}; also \textbf{LMS review}
\end{dframe}

\begin{dframe}{General web technology evolution}
  \1 Typically \textbf{don't need PHP} for simple content hosting

  \2 Dynamic web languages need server execution
  
  \3 Security and maintenance hassle...

  \1 Now common to aim to \textbf{generate static content}

  \2 Works well with content delivery networks (CDNs)
  \2 JavaScript is typically assumed for client-side Automation
  \2 Can include \textbf{dynamic content using \texttt{iframe}s}

  \1 Need something more powerful?
  
  \2 Grab yourself a virtual machine, or \textbf{a container}
\end{dframe}

\begin{dframe}{SSG---Static Site Generaors}
  \1 Move \textbf{application of templates} back to `compile time'

  \2 Run SSG over source code, HTML files produced
  \2 Often Markdown used for source code
  \2 Causes redundant HTML, but not a resource challenge

  \1 Popularised through \textbf{GitHub pages}

  \2 Web serving becomes simple and easy too

  % \3 ... which is why GitHub can do it for free

  \1 Many SSG tools exist, for example:

  \2 \textbf{\href{http://jekyllrb.com}{Jekyll}}---written in Ruby; used by GitHub Pages

  \2 \textbf{Hugo}---written in Go
\end{dframe}

\begin{dframe}{CI---Continuous Integration}
  \1 SSG tools work well with \textbf{continuous integration}, \emph{e.g.}, 

  \2 Update your source content;
  \2 Git commit, Git push;
  \2 CI runs SSG to build HTML;
  \2 HTML passed to webserver
   
  \bigskip
  
  \1 Let's \href{https://altitude.otago.ac.nz/cosc312}{create the website} for a new COSC paper
  
  \2 We'll use Hugo for the static site generation
  \2 \textbf{CS GitLab} at \href{https://altitude.otago.ac.nz/}{\texttt{altitude.otago.ac.nz}} provides \textbf{CI and web server}
 
  \3 Use of GitLab Pages covered in \href{http://cosc212.cspages.otago.ac.nz/LabBook/LabBook.pdf}{COSC212} \& \href{http://cosc202.cspages.otago.ac.nz/lab-book/COSC202LabBook.pdf}{COSC202} lab books
\end{dframe}

\begin{dframe}{Jekyll builds Carpentries teaching material}
  \1 The Carpentries promotes \textbf{digital research skills}

  \2 Releases teaching materials as open source via the web
  \2 Uses Jekyll SSG to produce its pages (\emph{e.g.}, GitHub Pages)

  \bigskip

  \1 Carpentries lessons used within COSC202 \& COSC349
  
  \2 \textbf{Unix Shell and Git} are core lessons
  \2 \textbf{Docker} lesson is in the Carpentries Incubator
  
  \bigskip

  \1 Let's build a \textbf{Carpentries lesson website} using Docker
\end{dframe}

\begin{dframe}{Visual Studio Code (VSCode)}
  \1 Wanted to start COSC202 on an \textbf{IDE usable broadly}

  \smallskip

  \1 \textbf{Chose VSCode}: development fast and productive
  
  \2 compositionality of extensions impressive---good design
  
  \3 \emph{e.g.}, I use a code spellcheck extension to spellcheck my \LaTeX{}

  \2 Integrates version control straightforwardly
  
  \3 thus supports Git-driven continuous integration

  \smallskip

  \1 Features for \textbf{future teaching exploration} include:

  \2 Support for developing within Docker containers
  \2 Support for developing on the cloud
\end{dframe}

\begin{dframe}{VSCode and Electron applications}
  \1 VSCode is an Electron application

  \2 Electron applications use \textbf{web tech for desktop apps}
  \2 Also means they're close to hosting on web

  \bigskip

  \1 VSCode evolved to be able to \textbf{run in web browser}

  \2 Server component runs actual VSCode backend
    
  \bigskip

  \1 VSCode's server component can run anywhere

  \2 \emph{e.g.}, your VSCode can be \textbf{running on a cloud server}

\end{dframe}

\begin{dframe}{GitHub Codespaces: a cloud-based IDE}
  \1 GitHub Codespaces beta launched around May 2020

  \2 \textbf{Pay-as-you-go IDE} in the cloud
  \2 Manage development environments \textbf{linked to repositories}
  \2 Active rather than the batch job feel of continuous integration
  \2 \dots but can commit back to repositories to trigger CI

  \bigskip

  \1 Let's use Codespaces to edit Carpentries lesson
\end{dframe}

\begin{dframe}{Teaching on BYOD and remote}
  \1 Advantages of cloud-based IDE tech include:
  
  \2 Can be hosted in the \textbf{cloud, onsite, or both}
  \2 Students can access the environment \textbf{on any device}
  \2 CS Lab environment or remote: consistent access to software

  \medskip

  \1 Challenge based on current papers: \textbf{use of GUIs}
  
  \2 Not impossible to handle, but typically web is targetted
  \2 Move to Electron development for GUI apps?
%  \2 and that doesn't suit all cases

  \medskip

  \1 May couple with CS lab shift to Windows for 2023
\end{dframe}

\begin{dframe}{Containers that build \LaTeX{} into PDF}
  % \1 Static websites for openly sharing open source lecture material is a bit niche

  \1 Using \LaTeX{} to make your lecture notes?
  
  \2 Can use \textbf{CI to share with students} the PDFs produced
  \2 Need a Docker container that builds \LaTeX{}
  
  \medskip
  
  \1 Using VSCode to edit your \LaTeX{}?
  
  \2 Can use online VSCode to \textbf{update your notes from anywhere}
  
  % \2 Automation efficiency: want compilation of notes to release them on the web too
  
  \medskip

  \1 Use links to share PDFs from Blackboard
 
  \2 i.e., URLs from Blackboard \textbf{direct to CI-generated PDFs}.

  \2 (or just link out to your own static website...)

  % \1 Discuss creating a container that builds content using \LaTeX{}
\end{dframe}

% include SOAP?

\begin{dframe}{Gitpod: open-source cloud IDE}
  \1 GitHub Codespaces is great, but \textbf{closed technology}

  \bigskip
  \1 Gitpod is to Codespaces what GitLab is to GitHub
  
  \2 \textbf{Open source implementation} of cloud IDE

  
  \2 Gitpod also provides a hosted service in the cloud
  
  \bigskip
  \1 Originally Gitpod used a VSCode-like editor
  
  \2 however, now `everyone' uses VSCode, so Gitpod does too
  \2 \textbf{Gitpod shared OpenVSCodeServer}---open VSCode back-end
\end{dframe}

\begin{dframe}{Git-based web can share large files (LFS)}
  \1 Many ways to \textbf{share files publicly} of different sizes

  \2 \emph{e.g.}, CloudStor, OneDrive(ish), FileDrop, Globus, etc.
  \2 (ITS can advise on the different options)
%  \2 sharing seems to only be for limited times only

  \1 CS website allowed \textbf{persistent public sharing} of files

  \2 but adding transient files to Git not ideal
  \2 `deleted' files will still be in Git history

  \1 \textbf{GitLab supports Git LFS}---i.e., large file support

  \2 Files in Git are instead more like symlinks
  \2 Git LFS organises for large files to be downloaded

  % example of using LFS to share files
\end{dframe}

\begin{dframe}{Takeaways}
  \1 Cloud-based software development \textbf{evolving quickly}

  \2 Worth watching for \textbf{teaching-relevant opportunities}

  \medskip
  
  \1 Discussed pros: \textbf{unify BYOD, lab and remote teaching}

  \medskip
  
  \1 Also challenges: graphics use, local hardware devices

  \medskip
  
  \1 Keen to \textbf{host Gitpod locally} for prototyping:
  
  \2 test tighter \textbf{teaching software integrations}
  \2 test \textbf{porting software} we already use to cloud IDEs

\end{dframe}

\begin{comment}
\begin{dframe}{FaaS, serverless and responsiveness}
  \1 Edge appearance
\end{dframe}
\end{comment}

\end{document}
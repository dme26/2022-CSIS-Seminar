\documentclass[aspectratio=169,t]{beamer}
\usepackage[utf8]{inputenc}
\usepackage[T1]{fontenc}
\usepackage{comment}
\usepackage{listings}
% \lstset{basicstyle=\scriptsize}

\title[2022-05-27---CSIS Seminar---David Eyers]{More cloudy student teaching---now with IDEs}
\date{CSIS Seminar, 2022-05-27}
\author{\textbf{David Eyers}}

\usetheme{oucs1}

\begin{document}

\begin{frame}
  \titlepage
\end{frame}

\begin{dframe}{Motivation and plan}
  \1 Again talking about cloud potential for teaching\\
     \dots but lots has changed in the last two years, e.g.,

  \2 I now edit \LaTeX{} using Visual Studio Code instead of Emacs
  \2 I finally converted Lech's slide template to Beamer
  
  \bigskip

  \1 My plan is to cover two topics:

  \2 Using continuous integration for teaching material
  \2 Potentially adopting cloud-based IDEs for teaching

  \bigskip

  \1 Also aim to explain relevance of two changes above!

\end{dframe}

\begin{dframe}{Prism---the CS webserver(ish)}
  \1 CS course material was hosted on \url{cs.otago.ac.nz}

  \2 Content normally reaches the web care of \url{prism.otago.ac.nz}
  \2 Supports dynamic pages, such as PHP
  \2 Security isolation through server design

  \bigskip

  \1 Pandemic shifted many papers to Blackboard

  \2 Blackboard handles some student/teaching needs OK?
  \2 Blackboard will be moving to the cloud
  \2 There's also an upcoming LMS review
  \2 BAD: Content is siloed!
\end{dframe}

\begin{dframe}{Web technology evolution}
  \1 Typically don't need PHP for simple content hosting
  
  \2 Common to now have a spectrum:
  \2 Static sites
  \2 Dynamic content

  \1 Can often link the two together, e.g., using iframes
  \1 Need something powerful? Grab yourself a virtual machine
\end{dframe}

\begin{dframe}{SSG---Static Site Generaors}
  \1 Pull templating back to compile time
  \1 Popularised through GitHub pages
  \1 Many SSG tools exist

  \2 Jekyll
  \2 Hugo
\end{dframe}

\begin{dframe}{CI---Continuous Integration}
  \1 SSG tools work well with continuous integration
  \1 For example, let's create the website for a new paper
  % template creation of COSC312
\end{dframe}

\begin{dframe}{Git LFS}
  \1 Prism allowed one to plonk large files online for sharing
  \1 OneDrive / Sharepoint is supposed to support sharing

  \2 It does, provided that you have narrow, corporate definitions
  \2 i.e., content is an office document or near to It
  \2 sharing is for limited times only---nothing should be too open

  \1 ITS can help in many ways, but we're not at self-admin levels yet

  \1 Sharing through GitLab Pages feels wrong---objects shouldn't be in Git
  
  \1 Enter LFS!
  % example of using LFS to share files

\end{dframe}

\begin{dframe}{Containers with \LaTeX}
  \1 
\end{dframe}

\begin{dframe}{VSCode + \LaTeX}
  \1 
\end{dframe}

\begin{comment}
\begin{dframe}{Codespaces}
  \1 
\end{dframe}

\begin{dframe}{Gitpod}
  \1 
\end{dframe}

\begin{dframe}{BYOD and remote teaching}
  \1 
\end{dframe}

\begin{dframe}{FaaS, serverless and responsiveness}
  \1 Edge appearance
\end{dframe}
\end{comment}

\end{document}
